% Time-stamp: <2024-04-26 09:08:32 vladimir>
% Copyright (C) 2019-2024 Vladimir G. Ivanović
% Author: Vladimir G. Ivanović <vladimir.ivanovic@sjsu.edu>
% ORCID: https://orcid.org/0000-0002-7802-7970
% arara: lualatex

% Time-stamp: <2024-04-25 09:53:14 vladimir>
% Copyright (C) 2019-2024 Vladimir G. Ivanović
% Author: Vladimir G. Ivanović <vladimir.ivanovic@sjsu.edu>
% ORCID: https://orcid.org/0000-0002-7802-7970
% arara: lualatex

\documentclass[article,letterpaper,12pt,twoside,no-math]{memoir}

\usepackage[USenglish]{babel}
\usepackage[style=apa,sortcites=false,sorting=nyt,backend=biber]{biblatex}
\usepackage{csquotes}
\usepackage{enumitem}
\usepackage{fontspec}
\usepackage{framed}
\usepackage{fullpage}
\usepackage{grffile}            % allows spaces in file names
\usepackage[american]{isodate}
\usepackage{multicol}
\usepackage{multirow}
\usepackage{mVersion}
\usepackage{orcidlink}
\usepackage{titling}
\usepackage{url}
\usepackage[svgnames]{xcolor}

\usepackage{hyperref}           % Must be last package loaded

% Set fonts
\setmainfont[Ligatures=TeX,Numbers=OldStyle,Numbers=Monospaced]{Alegreya}
\setsansfont[Ligatures=TeX,Numbers=OldStyle,Numbers=Monospaced]{Alegreya Sans}
\setmonofont[Scale=0.8]{Fira Mono}

\setmathrm[Numbers=Monospaced]{Alegreya}
\setmathsf[Numbers=Monospaced]{Alegreya Sans}
\setmathtt[Scale=0.8]{Fira Mono}

\raggedright                    % SJSU EDD Guidelines Checklist #2, #4
\raggedbottom

\def\UrlFont{\footnotesize\texttt}
\urlstyle{tt}

\setsecheadstyle{\Large\bfseries\raggedright\sffamily}
\setsubsecheadstyle{\large\bfseries\raggedright\sffamily}
\setlength{\parindent}{2em}
\setlength{\headheight}{15pt}

\cleanlookdateon% Remove ordinal day reference, e.g. st, nd, th

\hypersetup{%
  pdfinfo={%
    Title={Rocketship Education: An Study in Unintended Consequences},
    Author={Vladimir G. Ivanović},
    Language={en-US},
    Subject={},
    Keywords={Rocketship Education, education public policy},
    Version={\version}
  }
}

% \setVersion{0.1}                % Major.minor.build; Increment major and minor as appropriate
\increaseBuild{}

%%% Local Variables:
%%% mode: latex
%%% TeX-master: "Rocketship_Education__A_Study_of_Unintended_Consequences"
%%% End:


\addbibresource{References.bib}

\begin{document}

\pagestyle{plain}

\SingleSpacing%
\title{\vspace{-10ex}Rocketship Education: \\A Study of Unintended Consequences}
\author{Vladimir G. Ivanović}
\maketitle

\DoubleSpacing%

\section*{Introduction}
\label{sec:introduction}

Rocketship Education is a non-profit charter management organization (CMO) of 21 individual charter schools located across the United States, 13 of which are in California. Rocketship, founded in 2006, opened its first school in Santa Clara County, California in 2008. It pioneered an educational and financial model that is successful, has been a model for other CMOs, and offers lessons in unintended consequences.

Charter schools in California were created by the Charter Schools Act of 1992. The stated intent was to improve pupil learning, especially for underserved children, encourage both new teaching methods and professional development for teachers, provide parents and students with expanded opportunities to choose a school that would best meet their needs, move to a performance-based accountability system that focused on measurable outcomes, and finally provide \textquote{vigorous competition} with public schools. The act also exempted charter schools from almost all of the Education Code that traditional public schools (TPSs) are required to adhere to. Charters, after submitting a petition, would be authorized for a limited time (five years) after which they had to reapply periodically for authorization.

The Legislature's intentions all had unintended consequences, some of which have been exploited by Rocketship in ways that the Legislature likely did not foresee. But even before Rocketship was founded, the Charter Schools Act of 1992 was itself an unintended consequence of the ideas of the godfathers of charter schools, Ray Budde and Albert Shanker.

Budde and Shanker envisioned groups of \textit{teachers} creating a school within a school, one where teachers could experiment on how best to meet the educational needs of their students, relegating the school district to purely administrative functions. And although parental choice was mentioned, there was no notion of charter schools providing vigorous competition with anybody, much less with TPSs. That idea was introduced by Milton Friedman

Somehow Budde and Shanker's vision morphed into schools divorced from existing public school districts, with accountability systems that measured outcomes based on standardized tests which were coupled with a perform-or-shutdown mentality \parencite[5]{EdSource2004}. Teachers as drivers and owners of change receded from view and instead the focus became real estate and finance, not at all what Budde and Shanker had envisioned.

The remainder of this chapter will look at three aspects of charter school law in California: competition, facilities, and flexibility. These aspects have heavily influenced how Rocketship is structured and how it operates. For each aspect, this chapter will examine the law and its intent, and then how that intent produced unintended consequences.

\section*{Competition}
\label{sec:competition}

California Education Code §47600 et seq.\footnote{\url{https://leginfo.legislature.ca.gov/faces/codes_displayText.xhtml?division=4.&chapter=1.&part=26.8.&lawCode=EDC&title=2.}} specifies eight goals for charter school. The eigthth goal, added in 1998, is \textquote{(g) Provide vigorous competition within the public school system to stimulate continual improvements in all public schools}. The intent of this goal is clear. Yet the state has never measured if charter schools do in fact stimulate improvement in public schools. The word ``competition'' rarely appears in analyses by the Legislative Analyst's Office (LAO) or in policy briefs. A mention was made in a review of charters schools in Sonoma, CA by the \citeauthor{SuperiorCourtOfSonoma2003} which said that just a few charter schools cooperated with TPSs \parencite[3]{SuperiorCourtOfSonoma2003}. In 2005, a RAND study by \citeauthor{Zimmer.Buddin2009} concluded that \textquote[{\parencite[5]{Zimmer.Buddin2009}}]{charter schools are not creating competitive effects in California}.

One might expect that with so little attention paid to vigorous competition that that goal had little effect on charter schools. But such is not the case. It has at least three consequential unintended effects. Charter schools are constantly in search of students, preferably the right kind of students. Thus they need to be able to market themselves effectively to attract students, and finally, this competition for students makes charter schools view TPSs as the enemy.

\subsection*{The Search of Students}
\label{sec:search-for-students}

Unlike TPSs which have ready access to a supply of students, i.e. those which reside within district boundaries, charter schools must convince parents that their children will learn more by going to a charter school rather than gong to a TPS. Public schools need to do nothing and students come to them. Charter schools need to fight for every student in the face of a well-know alternative: the local public school. Sometimes attracting students is easy because the local TPS has been systematically underfunded for decades, is located in an area with significant crime and high unemployment, and parents are understandably desperate. At other times, it takes slick marketing, artificially inflated results, and promises that will not be met to convince a parent to send their child to a charter school.

Rocketship's marketing is well done. Their web site landing page immediately conveys three messages: you are not alone, your child will do exceptionally well at a Rocketship school, and it's free. First, the web site says that thousands of students (30,499) have attended Rocketship schools, so you, the parent, are not alone in choosing Rocketship for your child. Secondly, your child will be on a learning rocketship. Third and finally, not to worry, everyone can afford to go to a Rocketship school because it is free!

It's noteworthy that every child portrayed on the Rocketship web site is smiling and is wildly enthusiastic.  They all wear Rocketship purple shirts which immediately identifies a Rocketship student. Wearing clothes that stand out and are shared with every other student is a powerful generator of belonging\parencite{Iannaccone1994}.

Rocketship, whose explicit goal is to close the achievement gap, claims on their web site, that their students \textquote[{\parencite{RSED2023}}]{learn 5-7 \textit{months} [emphasis added] more every school year}. This is an exaggeration because it is not what is reported in \textcite{Raymond.etal2023}, the study they refer to. \textcite[132]{Raymond.etal2023} show that Rocketship's estimated annual growth in reading and math to be 0.166 and 0.239 respectively. For a standard 180 day year (in California), this is just under 30 days growth in reading and just over 43 days in math, neither are anywhere near the 5 to 7 months claimed, and it certainly doesn't apply cumulatively to the six years that comprise K-5. However, on the plus side, a SRI Education report, \textcite{Tyler.etal2016}, not peer reviewed, concluded that Rocketship students outperformed their peers in middle school by roughly the amounts claimed on Rocketship's web site, but just for the first year of middle school.

Rocketship's need for a robust pipeline of students is ever present, and so we see rather loose interpretations of reports and claims that go beyond what the evidence supports, all in an effort to maintain this pipeline of students. It ought to be mentioned that charter schools are sometimes picky about what students they'd like to see enrolled. \citeauthor{Mommandi.Welner2021a} summarize their book \citetitle{Mommandi.Welner2021} on how charter schools influence which students enroll. They have placed they ways \textquote{how charter schools control access and shape enrollment} into 13 general categories. Rocketship, for example, only accepts students in grades K–5. This might fall into the first of their categories, \textit{Description and Design: Which Niche?}, where \textquote{charter schools' description and design may influence applicant pools by indicating the students they seek to serve}.\footnote{Limiting grade levels is not illegal in any way, but it is puzzling that Rocketship doesn't have a middle school or a high school. It is speculated that since high school graduation is seven years into the future, limiting the grades served to K-5 allows Rocketship to avoid being accountable for the outcomes that really matter: Is a student prepared for college or career?} \citeauthor{Leung.etal2016} cover the same ground as \citeauthor{Mommandi.Welner2021}, but restrict themselves to practices that are strictly illegal.

Charter schools' search for students is an unintended consequence of charter school law. Although the law deliberately intended that there be competition, there was no recognition that charter schools would in any way spend time enticing students to enroll. The expectation seems to be that students will magically present themselves for enrollment without the need for extensive marketing, a subject covered in the next section.

\subsection*{The Need for Marketing}
\label{sec:need-for-marketing}

One requirement that charter schools need to meet before being allowed to open is to circulate a petition to obtain the signatures of parents \textquote{meaningfully interested} in having their child attend that charter school. This is the first marketing effort that charter schools undertake and it is usually a low key affair. But, even before a single student enrolls, the ground has been prepared by sophisticated marketing at the national and local levels.

The first exposure for most people are the names chosen for charter schools and related entities, for example:    Rocketship, New America, Knowledge is Power Program, High Tech High, IndySTEAM, Uncommon Schools, and one which makes two impressions at once, Aspire Public Schools. Names are critically important because they are repeated over and over, and they assume an outsized influence on the perception of a charter school or charter management organization (CMO). Would you rather your child go to a charter named after your city or to Aspire Public Schools?

Nationally, many markteting firms specialize in charter schools. Google reports 7,630,000 hits for the search string ``charter school marketing firms''.\footnote{It's doubtful that there are over 7 million different charter school marketing firms. That number is just the number of web sites which have the string ``charter school marketing firm'' in them somewhere.} There are also dozens of charter school advocacy organizations producing social media postings, blog articles, and reports which promote charter schools. All of them claim to be unbiased, with only the welfare of children in mind.  An example is The74\footnote{\url{https://www.the74million.org/}}, a reference to the 74 million children who were in public schools in the U.S. at the time The74 was founded. A glance at who supports them shows many of hugely wealthy foundations known to be charter school supporters: the Bill \& Melinda Gates Foundation, the Chan Zuckerberg Initiative, the Walton Family Foundation.\footnote{The74, owned by The 74 Media, Inc. is doing quite well. In 2022 they received nearly \$4.5M in donations and had just over \$3.7M in expenses.} For a more in depth treatment of charter school marketing, see \citeauthor{Lamberti2015} who dissects charter school marketing in Chicago. He says, \blockquote[{\parencite[2]{Lamberti2015}}]{It’s clear that charters use marketing to create customers for their schools not by helping parents make rational choices, but by appealing to values through messages that make charters feel like fresher, better alternatives to traditional public schools}.

Locally, Rocketship makes heavy use of its web site to promote itself, but relies on Innovate Public Schools\footnote{\url{https://innovateschools.org/}} to spread its message to the wider community.\footnote{Innovate Public Schools is surprisingly larger than The74. Expenses were just under \$7.7M and revenue was just shy of \$8M in 2022.} Innovate has been relentless in promoting charters. They hit upon rating schools, TPSs and charters, on how well low-income African-American and Latino perform on standardized tests, an unsurprisingly, charter schools including several Rocketship schools do well. These results are interesting if you value test taking ability, but they are meaningless if you are interested in say, the 6 C's of 21\textsuperscript{st} century education: communication, collaboration, critical thinking, creativity, citizenship, and character.

In any case, the results reported by the Santa Clara County Office of Education are wildly different that those reported by Innovate Public Schools. In 2021-22, just a few Santa Clara County authorized Rocketship charter schools did better than the state average on the Smarter Balanced Assessment Consortium (SBAC) English Language Arts (ELA) tests that are part of the California Assessment of Student Performance and Progress (CASPP) and none did better than the average for all Santa Clara County schools. The results for Mathematics (Math) are better, but still not stellar. These discrepancies are worthy of future investigation.

In the meantime, it should be noted that charter schools like Rocketship spend considerably more than the Legislature likely intended on marketing. The Legislature probably didn't expect that any money would be spent on marketing; there is no mention of the need in any document that recorded concerns about the original bill (SB 1448 (Hart)).\footnote{See, for instance, the documents collected by The National Charter School Founders Library that have to do with California in 1992: \url{https://charterlibrary.org/search/?_state=ca&_year=1992}}

\subsection*{Public Schools as the Enemy}
\label{sec:public-schools-as-enemy}

Competition has one other significant unintended consequence that is actually the opposite of what the Legislature intended: charter schools view public schools as the enemy and devote considerable effort into convincing the public that TPSs are bloated bureaucracies of sclerotic teachers resistant to any change with ineffective and bureaucratic administrators more concerned with job safety than with educating children. There also seems to be a visceral aversion to unions, probably because teachers are education professionals who know more about how to teach  than charter school operators.

One thing about competition is clear: competition produces winners and losers, and charter schools do no want to be losers. Hence they they try to daemonize TPSs. One example of this trend is a report from Innovate that implicates unidentified schools as both over-identifying and under-identifying children of color as having special needs \parencite{Winston2019}. Left unsaid  is that TPSs are the ones doing this and Innovate is merely  attention to this failure.

The Legislature intended for charter schools to innovate and then to pass those innovations on to TPSs. But if every student who attends a TPS is a student who does not attend a charter school, why would a charter school in their right mind consider helping to improve a TPS? Instead of cooperation, one is much more likely to have to explain to a parent that the failures of special education are much more likely due to it being only 40\% funded. Also unmentioned is the well-documented fact (which can easily be verified by comparing a charter school to the public school district in which it is located) that charter schools are loathe to admit students with special needs, and if they do, they admit fewer students with moderate to severe disabilities. Rocketship, for example, outsources its special education to a SELPA located in the Sierra foothills.

It is indeed unfortunate that charter schools view TPSs as the enemy, surely an unintended consequence of Milton Friedman's insistence that competition solves all problems.

\section*{Facilities\footnote{Substantial portions of this section were taken from the author's doctoral dissertation.}}
\label{sec:facilities}

Facilities are especially complicated for charter schools and even more so for Rocketship. There are variations in the kinds of facilities, how they are paid for, and what the relationship is between the entities involved. Facilities have an outsized impact on Rocketship because those issues are the primary driver of Rocketship's corporate structure.

\subsection*{The Choices for Facilities}
\label{sec:choices-for-facilities}

As shown in \prettyref{tab:charter-facilities-options}, charter schools have three options: co-locate, lease, or purchase (with or without construction of bespoke facilities).

\begin{table}[ht]
  \small%
  \caption[Charter School Facilities Options]{\textit{Charter School Facilities Options}}\label{tab:charter-facilities-options}%
  \begin{tabular}{ll}
    \toprule%
    Option    & Description \\
    \midrule%If the school's facilities are leased, and SB740 funds are used to pay part of the rent, then appraisals and the amount of rent should be available from the administator of the SB740 program.
    Co-locate & \multirow[t]{2}{4.75in}{The charter school occupies ``reasonably equivalent'' facilities provided by 
                the public school district in which the charter school is located.}\\
                \\
    Lease     & The charter school occupies facilities that it leases.\\
    Own       & The charter school buys existing facilities or buys land and builds their own. \\
    \bottomrule%
  \end{tabular}
\end{table}

The least costly option for charter schools is to co-locate in an existing school. Proposition 39 and enabling regulations\footnote{Ed. Code §47614 et seq.  and 5 CCR § 11969.1} require that school districts furnish facilities for all in-district charter school students that are reasonably equivalent to those of students in the district in which the charter school resides. Facilities include regular and specialized classrooms, administrative offices, playgrounds, and athletic fields. It does not matter if the school district has unused space or not. It does not matter if the charter school grows in enrollment year over year. The requirement is rigid; school districts are required to furnish reasonably equivalent facilities under Proposition 39. However, districts and charter schools may enter into agreements outside of Proposition 39 concerning what facilities districts will provide to the charter school.

Charter schools may lease their facilities from either a related party such as a 509(a)(3) supporting charity, or at arms length, from an unrelated party. The terms and length of leases vary. If the lessor is an unrelated party, the charter schools may take advantage of grants offered by the Charter School Finance Authority\footnote{Originally administered by the Department of Education; now administered by the California State Treasurer.} which are authorized by Ed. Code §47614.5 et seq. and CCR §10170\footnote{aka SB740} ``to offset annual on-going facility costs for charter schools that service a high-percentage of students eligible for free or reduced-price meals (FRPM) or located in a public elementary school boundary serving a similar demographic'' \parencite{CATreasurer2023}. The amount of the grant is the lesser of the school's ADA × \$1,420 or the annual rent × 75\%.\footnote{This is the basic calculation. As expected, there are variations and permutations, and these are enumerated in §6, Grant Award Calculations of the program's FAQ\@.} To be eligible, 55\% of a charter school's students must be eligible for free or reduced-price meals (FRPM) or be located in a public elementary school boundary serving a similar demographic \parencite[§1]{CATreasurer2023}.

The third way of obtaining facilities is to own the needed facilities, or to have a related party own the facilities. These might be purchased, or the land purchased and facilities constructed. Most public school districts own their own facilities, but since these were likely bought and built using bond money derived from taxes, charter schools, lacking taxing authority, are unable to pay for their facilities this way.

\subsection*{Source of Funds}
\label{sec:source-of-funds}

If a charter school's leadership decides they should own their own facilities, there are a number of ways they can go about this, as shown in \prettyref{tab:paying-for-facilities}.

\begin{table}[ht]
  \caption[Charter School Options for Paying for Facilities]{\textit{Charter School Options for Paying for Facilities}}%
  \label{tab:paying-for-facilities}%
  \begin{tabular}{ll}
    \toprule%
    \textbf{Option}    & \textbf{Source of Funds} \\
    \midrule%
    \protect\medskip%
    \multirow[t]{2}{1.25in}{Private grants or loans} & \multirow[t]{2}{4.25in}{Private entities (individuals or foundations) may make a grant or a loan to  a charter school.}\\\\ %
    \protect\medskip%
    Venture funds & \multirow[t]{2}{4.25in}{Venture Funds which ostensibly intend to make money often loan money to charter schools.}\\\\
    \protect\medskip%
    \multirow[t]{3}{1.25in}{Federal or state grants} & \multirow[t]{3}{4.25in}{Both the federal government and states have programs which offer funds that may be used to pay for existing facilities or for new construction.}\\\\\\
    \protect\medskip%
    Tax credits & \multirow[t]{2}{4.25in}{The federal government offers tax credits for investors whose investments meet
certain criteria.}\\\\
    Bonds & \multirow[t]{3}{4.25in}{Charter schools may use the commercial or municipal bond markets to obtain funds, but property or parcel taxes may not be used to pay them off.}\\\\\\
    \bottomrule%
  \end{tabular}
\end{table}

\section*{Flexibility}
\label{sec:flexibility}

\subsection*{Blended Learning}
\label{sec:blended-learning}

\subsection*{Virtual Charter Schools}
\label{sec:virt-chart-scho}

\subsection*{Sweeps}
\label{sec:sweeps}

\subsection*{The Opportunity for Fraud}
\label{sec:opportunity-for-fraud}

\section*{Conclusion and Policy Recommendations}
\label{sec:conclusions-policy-recommendations}

\OnehalfSpacing%
\printbibliography{}

\end{document}
\bye

The first thing to note that competition implies winners and losers. TPSs are relatively immune to going out of business, unlike charter schools which, in theory, might not be able to renew their charter and thus go out of busineess. The possiblity of ceasing to exist has a remarkable ability to focus one's mind, and so charter schools are in a constant

What's interesting is that there a test of how well charter schools do overall: compare charter schools, even high performing charter schools, to private schools. Private schools live or die based on how well they educate the children of parents who can afford \$40K, \$50K, \$60K in tuition annually. Even more interesting, private schools don't market themselves nearly as extensively as charter schools do; they rely on word-of-mouth, or lists of the colleges their students have been accepted into. 

It's unlikely that any charter school can produce well-rounded, knowledgeable students who are prepared for college or career 



%%% Local Variables:
%%% mode: latex
%%% TeX-master: "Rocketship_Education__A_Study_of_Unintended_Consequences"
%%% End:
